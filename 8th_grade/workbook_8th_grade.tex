
\documentclass{article}
\usepackage[utf8]{inputenc}
\usepackage[bulgarian]{babel}

\usepackage{systeme}
\usepackage{amsmath}

\usepackage{amsthm}
\theoremstyle{plain}

\usepackage{cmap}
\usepackage[utf8]{inputenc}
\usepackage[T2A]{fontenc}

\newtheorem{definition}{Дефиниция}

\usepackage{comment}
%\use
\newtheorem*{problem*}{Задача}
\newtheorem{problem}{Задача}

\newtheorem{theorem}{Теорема}
\newcounter{solution}

\usepackage{graphicx}
\usepackage{pst-plot}

\usepackage{tikz}

\newcommand\solution{%
	\stepcounter{solution}%
	\textbf{Решение :}\\%
}

\usetikzlibrary{angles,
	quotes}
\usepackage{siunitx}


\date{}

\title{Книжка за упражнителни задачки 8ми клас}
\begin{document}
	
	
	\maketitle
	\tableofcontents
	
	\section{Начални бележки}
	Започваме с един Quiz:
	\begin{itemize}
		\item 	Има ли конкретен тип задачи, които искаш да ти покажа? Кой е той?
		\item Кое ти допада повече - алгебра или геометрия? Защо?
		\item Може ли да ми дадеш някаква част от задачите, които не си успял да решиш? Търсим типовете грешки, върху които трябва да поработим.
		\item Кое ти е било най-трудното от математиката за 8ми клас? Кое най-лесното?	

	\end{itemize}
	
	Допълнителни въпроси:
\begin{itemize}
	\item Искаш ли да ходиш на състезания по математика?
	\item Искаш ли да чуеш приложения на математиката, които не са в учебника?\\
	
\end{itemize}	
	
	Типове задачи, които можеш/не можеш да решаваш:
\begin{enumerate}
	\item квадратни уравнения - виет, корени $x^2 - 5x + 6 = 0 $
	\item дробни уравнения 
	\item задачи с вектори
	\item геометрия - задачи с трапец
	\item геометрия - задачи с окръжности
	\item геометрия - медицентър и средна отсечка
\end{enumerate}



\section{Вероятности}
	
	пример. Хвърляне на зар.\\
	Като хвърляме зар, имаме 6 възможности - 1,2,3,4,5,6 като всяка от тях има вероятност  %$	\fraction{1}{6} $ . 
	Вероятността да хвърлим два пъти 1 последователно е 1/36.\\
	пример. Двуцифрени числа, записани с цифрите 2,3,4,5\\
	
	Първата цифра може да е 2,3,4 или 5 и втората цифра също може да е 2,3,4,5.
	\\
	
 Двуцифрени числа, записани с цифрите 2,3,4,0\\
 Всички двуцифрени числа

пример с топки. Имаме урна с 10 топки, 5 червени, 3 сини, 2 бели. Каква е вероятността при 3 изтеглени топки(с или без връщане) да имаме:\\
I - 2 червени, 1 синя\\
II -  1 червена, 1 синя, 1 бяла



\section{Рационални изрази}
Естествени числа, Цели числа, Рационални числа, Ирационални числа.\\
Всяко рационално числа има представяне като безкрайна периодична десетична дроб.
Пример $\frac{1}{7} = 0.142857(142857) $ \\
Безкрайните непериодични десетични дроби се наричат ирационални числа.
Пример. $ \sqrt 2 \approx 1.41...$ . \\
всичките изброени числа по-горе образуват реалните числа.

Действия с рационални изрази:\\
Привеждане под общ знаменател: \\
$$\frac{x-1}{x^2-4} - \frac{3-x}{x-2} = \frac{x-1}{(x-2)(x+2)} + \frac{(x-3)(x+2)}{(x-2)(x+2)} =  $$
$$ = \frac{x-1 + x^2 - x - 6 }{(x-2)(x+2)}  $$

\begin{problem}
	$\frac{1+x}{x-1} - \frac{x-1}{1+x} = \frac{12}{1-x^2}$ \\
	$\frac{1+x}{x-1} - \frac{x-1}{1+x} = \frac{12}{(1-x)(1+x)}$ \\
	$\frac{1+x}{x-1} - \frac{x-1}{x+1} = \frac{-12}{(x-1)(x+1)}$ \\
    ДС:$x \neq 1  $, $x \neq -1 $. \\
    $(1+x)(x+1) - (x-1)(x-1) = -12 $\\
    $(x+1)^2 - (x-1)^2 = -12 $ \\
    $x^2 + 2x + 1 -(x^2 - 2x +1) = -12  $ \\
    $x^2 + 2x + 1 -x^2 + 2x -1 = -12 $ \\
    $4x = -12 $\\
    $ x = -3$ 
    
    
\end{problem}




\section{Quiz}








\end{document}	
	
	

