
\documentclass{article}
\usepackage[utf8]{inputenc}
\usepackage[bulgarian]{babel}

\usepackage{systeme}
\usepackage{amsmath}

\usepackage{cmap}
\usepackage[utf8]{inputenc}
\usepackage[T2A]{fontenc}

\newtheorem{definition}{Дефиниция}

\usepackage{comment}
%\use
\newtheorem{problem}{Задача}
\newtheorem{theorem}{Теорема}
\newcounter{solution}

\usepackage{graphicx}
\usepackage{pst-plot}

\usepackage{tikz}

\newcommand\solution{%
	\stepcounter{solution}%
	\textbf{Решение :}\\%
}

\usetikzlibrary{angles,
	quotes}
\usepackage{siunitx}


\date{}

\title{Книжка за упражнителни задачки на Илиянка}
\begin{document}
	
	
	\maketitle
	\tableofcontents
	
	\section{Начални бележки}
	Спъсък с неща, които искаме да се научат:
	\begin{enumerate}
		\item Еднакви триъгълници.
		\item Задачи с текст
		\item Решаване на триъгълник
		\item Геометрия
		
		
	\end{enumerate}
	План за работа:
	\begin{enumerate}
		\item Решаване на триъгълник
		\item 
	\end{enumerate}
	
	\begin{problem}
		$(-0,5-x)^2 = (-0,5)^2 + 2.0,5x + x^2$ заместваме $x = -\frac{1}{2} $ и получаваме
	 $(-0,5)^2 + 2.0,5.(-\frac{1}{2}) + (-\frac{1}{2})^2 = 0.25 -\frac{1}{2} + \frac{1}{4}= \frac{25}{100}-\frac{1}{2} + \frac{1}{4} = \frac{5}{20}-\frac{1}{2} + \frac{1}{4} =$ \\
	 $\frac{1}{4}-\frac{1}{2} + \frac{1}{4} = \frac{2}{4} - \frac{1}{2}= \frac{1}{2} - \frac{1}{2} = 0 $ 
\end{problem}
	
	\section{Еднакви триъгълници}
	В признаците за еднаквост на триъгълници винаги имаме 3 \underline{неща}. Под нещо разбираме страна или ъгъл. 
	\begin{enumerate}
		\item две страни и ъгъл между тях
		\item страна и два ъгъла
		\item три страни
	\end{enumerate}
	
	
	


\end{document}	
	
	

