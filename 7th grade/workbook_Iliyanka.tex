
\documentclass{article}
\usepackage[utf8]{inputenc}
\usepackage[bulgarian]{babel}

\usepackage{systeme}
\usepackage{amsmath}

\usepackage{cmap}
\usepackage[utf8]{inputenc}
\usepackage[T2A]{fontenc}

\newtheorem{definition}{Дефиниция}

\usepackage{comment}
%\use
\newtheorem{problem}{Задача}
\newtheorem{theorem}{Теорема}
\newcounter{solution}

\usepackage{graphicx}
\usepackage{pst-plot}

\usepackage{tikz}

\newcommand\solution{%
	\stepcounter{solution}%
	\textbf{Решение :}\\%
}

\usetikzlibrary{angles,
	quotes}
\usepackage{siunitx}


\date{}

\title{Книжка за упражнителни задачки на Илиянка}
\begin{document}
	
	
	\maketitle
	\tableofcontents
	
	\section{Начални бележки}
	Спъсък с неща, които искаме да се научат:
	\begin{enumerate}
		\item Еднакви триъгълници.
		\item Задачи с текст
		\item Решаване на триъгълник
		\item Геометрия
		
		
	\end{enumerate}
	План за работа:
	\begin{enumerate}
		\item Решаване на триъгълник
		\item 
	\end{enumerate}
	
	\begin{problem}
		$(-0,5-x)^2 = (-0,5)^2 + 2.0,5x + x^2$ заместваме $x = -\frac{1}{2} $ и получаваме
	 $(-0,5)^2 + 2.0,5.(-\frac{1}{2}) + (-\frac{1}{2})^2 = 0.25 -\frac{1}{2} + \frac{1}{4}= \frac{25}{100}-\frac{1}{2} + \frac{1}{4} = \frac{5}{20}-\frac{1}{2} + \frac{1}{4} =$ \\
	 $\frac{1}{4}-\frac{1}{2} + \frac{1}{4} = \frac{2}{4} - \frac{1}{2}= \frac{1}{2} - \frac{1}{2} = 0 $ 
\end{problem}
	\section{Учебен материал/Рубрика 10 минути = 3 урока/}
	\subsection{Алгебра}
	\subsubsection{Алгебра}
	\subsection{Геометрия}
	\subsubsection{Еднакви триъгълници}
	В признаците за еднаквост на триъгълници винаги имаме 3 \underline{неща}. Под \underline{нещо} разбираме страна или ъгъл. 
	\begin{enumerate}
		\item Iви признак - две страни и ъгъл между тях
		\item IIри признак - страна и два ъгъла
		\item IIIти признак - три страни
	\end{enumerate}
	Запомняне - трети признак е три страни, на първи и втори броят страни е разменен.
\section{Задачи от учебна тетрадка}
\subsection{Упражнения за 1ви признак за еднаквост на триъгълници }	

\begin{problem}
	\begin{enumerate}
		\item 	$\triangle A'OD \cong B'OC' по първи признак(AO =CO ) $
		\item $AO \neq BO$ 
		\item $AOD' \cong COB'$
		\item $DO \neq CO$
	\end{enumerate}

\end{problem}


\begin{problem}
	\begin{enumerate}
		\item 	$\triangle A'OD \cong B'OC' по първи признак(AO =CO ) $
		\item $AO \neq BO$ 
		\item $AOD' \cong COB'$
		\item $DO \neq CO$
	\end{enumerate}
	
\end{problem}


\begin{problem}
	\begin{enumerate}
		\item 	$\triangle A'OD \cong B'OC' по първи признак(AO =CO ) $
		\item $AO \neq BO$ 
		\item $AOD' \cong COB'$
		\item $DO \neq CO$
	\end{enumerate}
	
\end{problem}
запиши всички видове числа \\
уравнения и неравенства се вземат заедно \\
разграЙ заедно уравнения и неравенства \\
чертай цветни картинки, първо рисунки, после картинки\\
първо го вземаме ние, или първо в училище?\\
40 минути задачи, останалото е аз показвам разни неща\\
когато каза да решаваме друго, си помислих, че си притеснена\\
не трябва да решаваме задачи като си притеснена\\
по-добре се ядосвай, но не се притеснявай\\
изпитът е доста дълъг, ще трябва "стратегия" за решаване, за да може да се съсредоточиш и на последните задачи \\
кога ще направим първия експеримент с 5 задачи решени и описани от единия и другия?\\



\section{Теми от стари изпити}
\subsection{Възможни стратегии}
\begin{itemize}
	\item Заместване на отговорите в уравнението за намиране на корен
	\item Изключване на отговори
\end{itemize}
\subsection{Тема 2013}
\subsection{Тема 2014}
\subsection{Тема 2015}
\subsection{Тема 2016}
\subsection{Тема 2017}
\subsection{Тема 2018}
\subsection{Тема 2019}

\newpage
\section{Quiz}

\subsubsection{Quiz 1}
\begin{itemize}
	\item Колко градуса е един прав ъгъл?
	\item Колко са признаците са еднаквост на триъгълници и може ли да изброиш равните елементи?
	\item 
\end{itemize}


\subsubsection{Quiz 2}
\begin{itemize}
	\item Колко градуса е един прав ъгъл?
	\item Колко са признаците са еднаквост на триъгълници и може ли да изброиш равните елементи?
	\item 
\end{itemize}




\end{document}	
	
	

