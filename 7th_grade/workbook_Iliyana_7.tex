
\documentclass{article}
\usepackage[utf8]{inputenc}
\usepackage[bulgarian]{babel}

\usepackage{systeme}
\usepackage{amsmath}

\usepackage{amsthm}
\theoremstyle{plain}

\usepackage{cmap}
\usepackage[utf8]{inputenc}
\usepackage[T2A]{fontenc}

\newtheorem{definition}{Дефиниция}

\usepackage{comment}
%\use
\newtheorem*{problem*}{Задача}
\newtheorem{problem}{Задача}

\newtheorem{theorem}{Теорема}
\newcounter{solution}

\usepackage{graphicx}
\usepackage{pst-plot}

\usepackage{tikz}

\newcommand\solution{%
	\stepcounter{solution}%
	\textbf{Решение :}\\%
}

\usetikzlibrary{angles,
	quotes}
\usepackage{siunitx}


\date{}

\title{Книжка за упражнителни задачки на Илиянка}
\begin{document}
	
	
	\maketitle
	\tableofcontents
	
	\section{Начални бележки}
	Спъсък с неща, които искаме да се научат:
	\begin{enumerate}
		\item Еднакви триъгълници.
		\item Задачи с текст
		\item Решаване на триъгълник
		\item Геометрия
		
		
	\end{enumerate}
	План за работа:
	\begin{enumerate}
		\item Решаване на триъгълник
		\item Неравенства
	\end{enumerate}
	
	\begin{problem*}
		$(-0,5-x)^2 = (-0,5)^2 + 2.0,5x + x^2$ заместваме $x = -\frac{1}{2} $ и получаваме
	 $(-0,5)^2 + 2.0,5.(-\frac{1}{2}) + (-\frac{1}{2})^2 = 0.25 -\frac{1}{2} + \frac{1}{4}= \frac{25}{100}-\frac{1}{2} + \frac{1}{4} = \frac{5}{20}-\frac{1}{2} + \frac{1}{4} =$ \\
	 $\frac{1}{4}-\frac{1}{2} + \frac{1}{4} = \frac{2}{4} - \frac{1}{2}= \frac{1}{2} - \frac{1}{2} = 0 $ 
\end{problem*}
	\section{Учебен материал/Рубрика 10 минути = 3 урока/}
	\subsection{Алгебра}
	\subsubsection{Алгебра}
	Алгебрата в училище приблизително се състои от решаване на няколко вида уравнения и неравенства. Понякога се искат (напр.) целочислени решения, затова ще напишем по-долу видовете числа:
	\begin{itemize}
		\item \textbf{Естествени}: 1,2,3,4... (цели-положителни)	
		\item \textbf{Цели}: 0,-1,1,-2,2,-3,3.....
		\item \textbf{Рационални}: Всички дроби (с числител и знаменател  цяло число) или безкрайни периодични десетични дроби
		\item \textbf{Реални}: Всички безкрайни десетични дроби	
	\end{itemize}
	\subsubsection{Едночлени/Многочлени, действия и формули(в уч. Цели изрази)}
	Разкриване на скоби - всяко се умножава със всяко. Пример:\\
	$(1-2+3)(4-5) = 1.4 - 1.5 + 2.4 + 2.5 - 3.4 + 3.5$. \\
	Въпрос. Има ли в горния ред грешки?\\
	
	Формули за съкратено уножениe \\
	$(a+b)^2 = (a+b)(a+b)=a.a + a.b + b.a + b.b = a^2 + 2ab + b^2 $ \\
	$(a-b)^2 = (a-b)(a-b)=a.a - a.b - b.a + b.b = a^2 - 2ab + b^2 $\\
	$(a+b)(a-b)= a^2 - ab + ab - b^2 = a^2 - b^2 $\\	
	$(a+b)^3 $\\
	$(a-b)^3 $\\
	$a^3 + b^3 $\\
	$a^3 - b^3 $
	
	\subsubsection{Уравнения}
	Уравненията са задачи за намиране на неизвестно число $x$. За решаването на уравнения се използва разкриване на скоби(всяко се умножава със всяко), събиране на едночлени и прехвърляне от едната страна на равенството в другата(при което се сменя знака).
	Пример: \\
	$ x - 4 = 2x + 3$ \\
	$ x - 2x = 3 + 4 $ \\
	$ -x = 7 $ \\
	$ x = -7. $ 
	
	Уравнения с параметър
	\subsubsection{Неравенства}
	Решаването на неравенства прилича на това на решаването на уравнения. Единствените разлики са следните:
	\begin{enumerate}
		\item При умножаване на двете страни на неравенството с отрицателно числ(<0), знакът на неравенството се сменя: $-x<-1  \iff x> 1$.
		\item Накрая решението се записва с интервали: $x>1 \iff$ $x$ в $ (1,+\infty) $.
		Неравенствата с $\leq$ or $\geq$ се записват със средна скоба $x \geq1 \iff$ $x$ в $ \left[1,+\infty \right) $.
	\end{enumerate}
	
	
	Неравенства с параметър
	\subsubsection{Задачи с текст - смеси и сплави, работа, движение, капитал}
	\subsubsection{Вероятности}
	Примери за вероятности: при хвърляне на зар каква е вероятността да хвърлим 1? Всички възможности са 6 и са равновероятни и тогава вероятността е $\frac{1}{6} $.
	
	\subsection{Геометрия}
	\subsubsection{Еднакви триъгълници}
	В признаците за еднаквост на триъгълници винаги имаме 3 \underline{неща}. Под \underline{нещо} разбираме страна или ъгъл. 
	\begin{enumerate}
		\item Iви признак - две страни и ъгъл между тях
		\item IIри признак - страна и два ъгъла
		\item IIIти признак - три страни
	\end{enumerate}


Симетрала на AB - права, която пресича AB и  сключва прав ъгъл с нея \\
перпендикулярни прави - сключват прав ъгъл юю
	Запомняне - трети признак е три страни, на първи и втори броят страни е разменен.
\section{Задачи от учебна тетрадка}
\subsection{Упражнения за 1ви признак за еднаквост на триъгълници }	


\begin{problem*}
	\begin{enumerate}
		\item 	$\triangle A'OD \cong B'OC' по първи признак(AO =CO ) $
		\item $AO \neq BO$ 
		\item $AOD' \cong COB'$
		\item $DO \neq CO$
	\end{enumerate}

\end{problem*}


\noindent
(To be continued.....)\\
(To be continued.....)




\section{Теми от стари изпити}
\subsection{Възможни стратегии}
\begin{itemize}
	\item Заместване на отговорите в уравнението за намиране на корен
	\item Изключване на отговори
	\item (Информирано) налучкване - не е препоръчително
\end{itemize}
\subsection{Тема 2013}
\subsection{Тема 2014}
\subsection{Тема 2015}
\subsection{Тема 2016}
\subsection{Тема 2017}
\begin{problem*}
	Коя е стойността на израза $2(3-c)-c(c-2) $ при $c=-3$?\\
	Разкриваме скобите и събираме: $2.3 -2.c -c^2 -c(-2) = 6-2c - c^2 + 2c = -c^2 + 6 = -3$. Отг. Б$\left. \right)$
\end{problem*}
	\begin{problem*}
	Изразът $mx-2x -2y + my$ е тъждествено равен на израза:\\
	Групираме и извеждаме пред скоби: $mx+my -2x -2y = m(x+y)-2(x+y) = (m-2)(x+y) $. Отг. A$\left. \right)$ 
\end{problem*}
\begin{problem*}
	Коренът на уравнението $x(x+4)-x(x+3) = 5x + 1$ e:\\
	Прехвърляме всичко с $x$ отляво и всички числа отдясно:\\
	$x^2 +4x - x^2 -3x -5x = 1 $\\
	$-4x = 1 $, откъдето $x = -\frac{1}{4} $. Отг. Б$\left. \right)$ 
\end{problem*}
\begin{problem*}
	Решенията на неравенството $18 -6x \geq 0$ са: 
	$6x \leq 18 $ или $ x \leq 3 $
	Отг. A$\left. \right)$ 
\end{problem*}

\begin{problem*} Произведението на корените на уравнението $\left|x-5 \right| -5 = 1 $ e: \\
	Намираме решениеята на уравнението $\left|x-5 \right| = 6 $, получаваме двете решения $x-5 = 6 \to x_1 = 6+5=11 $ и $x-5=-6 \to x_2 = -1 $.
	Отг. Г$\left. \right)$ 
\end{problem*}
\begin{problem*}0a0a0a0a
	Отг. A$\left. \right)$ 
\end{problem*}
\begin{problem*}
	Отг. В$\left. \right)$ 
\end{problem*}
\begin{problem*}
	Отг. Б$\left. \right)$ 
\end{problem*}
\begin{problem*}
	Отг. Г$\left. \right)$ 
\end{problem*}



Задачи с пълно решение \\
\begin{problem}
дадени са многочлените: \\
$M = (-2+3x)^2 -(2x-3)(3x+3) -6 +3(1-x)(1+x) = $\\ $(-2)^2 + 2.(-2)3x + (3x)^2  -(6x^2 + 6x -9x -9) -6 +3(1-x^2) =$\\
$4 - 12x + 9x^2 - 6x^2-6x +9x+9 -6 + 3 -3x^2 = x^2(9-6-3)+x(-12-6+9) +10 = -9x+10 $ \\ to be continued...

$-7x+7 \leq 0 $ \\
$ 7 \leq 7x +0 $\\
$ 7x \geq 7 $ (можем да делим и умножаваме с числа $ \geq 0$) \\ 
$ x \geq 1 $ \\
(!) Можем да умножаваме и с числа $\leq 0 $, но тогава обръщаме знака на неравенството юю

Пример:
$-3x < 6$ \\
(уравнението се решава:$-3x = 6 \to x = \frac{6}{-3} = -2 $  ) \\
$-3x < 6$ $|:(-3)$ \\
$x > \frac{6}{-3} $ или $x>-2 $. Отг x в$ (-2,)$ \\
Когато знакът е $<$ или $>$, тогава скобите са $()$. Когато знакът е $\leq $ или $\geq $, тогава скобата в числото е $[] $. Безкрайността се пише като обърната $8$-ца $\infty $ винаги е с $() $


 \end{problem}


\begin{problem}
	$2x > 1 $ \\
	$x > \frac{1}{2} $\\
	x в $(\frac{1}{2}, + \infty) $\\
	
	
	$4x - 5 < -2x + 13 $\\
	$4x +2x < 13 +5 $\\
	$6x < 18 $\\
	$x < 3  $
	
	$ -2x -1 > -x -5 $ \\
	$-2x +x >  -5 +1  $\\
	(Коментар:$-2x + 1x = (1-2)x = -x$ ) \\
	
	$-x > -4 |.$( -1)\\
	$ x < 4 $ \\
	Отг x в $(-\infty , 4) $
	запиши обяснение включва и не включва със скобите.
\end{problem}


\subsection{Тема 2018}
\begin{problem*}
	В турнир по спортна стрелба участват х отбора. Във всеки отбор има по у момчета и 2 пъти
	по-малко момичета. С кой от следващите изрази може да се определи броят на играчите, които
	участват в турнира?
\end{problem*}
\solution Броят на момчетата е $y$ по условие и тогава броят на момичетата е $\frac{y}{2}$. Тогава един отбор има $y + \frac{y}{2}$ хора. Всички хора са $x(y+\frac{y}{2}) $ \\

задачи 6,7, 14, 15

\begin{problem*}
	пътници
\end{problem*}

\solution
$a )$ Колко процента са 230 от 500? (Жокер: $1\% = \frac{1}{100} $). Трябва да напишем  $\frac{230}{500}$ като проценти. $\frac{230}{5.100} = \frac{230}{5}\% = 46\%$.

b) $170 + 4x + x+ 230 = 500 $ \\
$ 400 + 5x = 500 $ \\ $5x = 100 $ \\
$x = \frac{100}{5} = 20  $
Хората, които отиват и се връщат с градски гранспор са $4x = 80 $. Хората, които се прибират с градски транспорт са $4x +70 = 80 + 70 = 150.$ Остава да напишем $\frac{150}{500} $ като процент. 
$30\% $


\subsection{Тема 2019}
.....



\subsection{упражнения за неравенства}

\begin{problem}
	$4x - 7 \leq 3 $\\
	
	Да решим задачата като уравнение:\\	
	$4x = 3 +7 $ \\
	$4x = 10 $ \\
	$x = \frac{10}{4} = \frac{5}{2} $\\
	
	Тогава за неравенстово имаме:\\
	... сметки\\
	...сметки
	Отг. $x\leq \frac{5}{2} $
	
\end{problem}

\begin{problem}
	$2(x-1) -4x > 3 - (x-7)$\\
	$2x - 2 - 4x > 3 -x -(-7) $\\
	$ 2x - 4x +x > 3 +7 +2 $\\
	$(2-4+1)x > 3+7+2$\\
	$ -x > 12$\\
	$ x < 12 $
	
\end{problem}

\begin{problem}
	$3(5-x) + 4(x-4) - 1 \geq -1 + 2(1-x)$\\
	$15 -3x + 4x - 16 -1 \geq -1 +2 - 2x  $\\
	$-3x +4x +2x \geq -1 + 2 - 15 +16 + 1 $\\
	$(-3+4+2)x \geq 1 +1 + 1 $\\
	$ 3x \geq 3  $\\
	Делим числото от дясната стана на числото пред х:\\	
	$ x \geq \frac{3}{3} $\\
	$x \geq 1 $\\
	x  в $\left[ 1, +\infty \right) $
	
	
	
упражнение\\
$-2+1 = 1-2 = -1 $\\
$-1 +2 =2 -1 = 1 $\\

	
	
	
\end{problem}



\newpage
\section{Quiz}

\subsubsection{Quiz 1}
\begin{itemize}
	\item Колко градуса е един прав ъгъл?
	\item Колко са признаците са еднаквост на триъгълници и може ли да изброиш равните елементи?
	
	
	\item Какво е симетрала?
\end{itemize}


\subsubsection{Quiz 2}
\begin{itemize}
	\item Колко градуса е един изправен ъгъл?
	\item Какво е приведение?
	\item Какво е 'просто число?'
	\item Нормалният вид на многочлен
\end{itemize}


\begin{problem}" Махане на модул"
	$|y| = y   $, когато $y\geq 0 $\\
	$|y| = -y  $, когато $y<0 $\\
	
	$\left| x -3 \right| = 4 $\\
	$ x - 3= -4 $ е същото като  $ 3-x = 4 $ и $x = -1 $\\
	$ x - 3 = 4 $ и  $x = 7 $\\
	
	Отг $x_1 = -1  $
\end{problem}



 

\section{Решени изпити}

\begin{itemize}
	\item 2017 - цялата алгебра от тест, 17 и 23
	\item 2018 - всички тестови и задачи с неравенства и уравнения от останалите 
	\item за самостоятелно решаване - МАЙ 2012
	\item 
	\item 
	\item 
	\item 
	\item 
	
\end{itemize}


\end{document}	
	
	

